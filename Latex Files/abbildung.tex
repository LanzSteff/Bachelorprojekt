\chapter*{Abbildungsverzeichnis}

\begin{tabbing}
	\hspace*{0.5cm}\=\hspace*{4.2cm}\=\hspace*{3cm}\=\hspace*{5cm}\= \kill
	\>{\bf Abbildung} \> {\bf } \> {\bf } \> {\bf Seite} \\ \ \\
	\>Abb. \ref{single}: SingleLayer Network \itshape(de.wikipedia.org)\upshape \> \> \> \pageref{single} \\ \ \\
	\>Abb. \ref{multi}: MultiLayer Network \itshape(de.wikipedia.org)\upshape \> \> \> \pageref{multi} \\ \ \\
	\>Abb. \ref{recurrent1}: Recurrent Network 1 \itshape(de.wikipedia.org)\upshape \> \> \> \pageref{recurrent1} \\ \ \\
	\>Abb. \ref{recurrent2}: Recurrent Network 2 \itshape(web.mst.edu)\upshape \> \> \> \pageref{recurrent2} \\ \ \\
	\>Abb. \ref{iomodel}: Input-Output Model \itshape(sciencedirect.com)\upshape \> \> \> \pageref{iomodel} \\ \ \\
	\>Abb. \ref{ssmodel}: State-Space Model \> \> \> \pageref{ssmodel} \\ \ \\
	\>Abb. \ref{elmanNet}: Elman Network\itshape(plato.stanford.edu)\upshape \> \> \> \pageref{elmanNet} \\ \ \\
	\>Abb. \ref{basicElman}: Basic Elman Network \> \> \> \pageref{basicElman} \\ \ \\
	\>Abb. \ref{arche}: Architektur \> \> \> \pageref{arche} \\ \ \\
	\>Abb. \ref{sigmoid}: Sigmoide \> \> \> \pageref{sigmoid} \\ \ \\
	\>Abb. \ref{result1}: System Identification \> \> \> \pageref{result1} \\ \ \\
	\>Abb. \ref{result2}: Fehlerdarstellung \> \> \> \pageref{result2} \\ \ \\
	\>Abb. \ref{function}: Funktion \> \> \> \pageref{function} \\ \ \\
	\>Abb. \ref{result3}: Approximation \> \> \> \pageref{result3} \\ \ \\
\end{tabbing}